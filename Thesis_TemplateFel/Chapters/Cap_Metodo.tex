% Chapter Template

\chapter{Método} % Main chapter title

\label{Cap_Exp} % Change X to a consecutive number; for referencing this chapter elsewhere, use \ref{Cap_Exp}

%----------------------------------------------------------------------------------------
%	SECTION 1
%----------------------------------------------------------------------------------------
\section{Participantes}

Se contó con la participación de 50 estudiantes de los primeros semestres de la carrera de Psicología en la Facultad de Psicología de la Universidad Nacional Autónoma de México. A cambio de su participación todos los estudiantes recibieron medio punto extra sobre la  calificación de uno de sus exámenes, y los participantes que obtuvieron mayores ganancias al final del juego recibieron adicionalmente otro medio punto extra.\\

\section{Diseño experimental}

Se realizaron 10 sesiones experimentales con 5 participantes diferentes en un aula sin distracciones externas. Al inicio de cada sesión se asignó aleatoriamente un rol a cada participante (A, B, C, D y E).\\

Las sesiones estuvieron compuestas de ocho periodos de p-beauty contest, divididos en dos subjuegos de cuatro periodos. En cada subjuego participaron solo tres personas: En el primer subjuego, los participantes A, B y C jugaron por cuatro periodos, mientras los participantes D y E esperaban en un aula diferente sin poder hablar entre ellos. Terminado el subjuego 1, los participantes B y C se retiraban y eran reemplazados para el subjuego 2 por los participantes D y E. En el segundo Subjuego, los participantes A, D y E jugaron por cuatro periodos, al final de los cuales se dio por terminado el experimento.\\

Además de elegir su número en el juego, se solicitó a cada participante que en cada periodo escribieran dos números que creyeran que estarían lo más cerca posible de los números elegidos por los otros dos jugadores.\\

\section{Procedimiento}

En cada sesión se citó a los 5 participantes en un aula donde se les leyó las instrucciones del experimento: Se les explicó cómo jugar p-beauty contest y se les dio la instrucción adicional de  intentar adivinar el número a elegir por los otros jugadores y registrar dos creencias por periodo.También se les explicó que la sesión se dividiría en dos subjuegos con tres participantes en cada uno. Finalmente se hizo de su conocimiento las ganancias que podrían acumular en caso de dar el número más cercano al número objetivo (6 puntosa divididos entre todos los ganadores), o bien, de acertar en la predicción de las tiradas de sus compañeros (1 punto por cada número acertado dentro de un rango de +/-5). Las instrucciones se anexan como Apéndice al final del presente documento.\\

Los participantes asignados mediante sorteo a los roles D y E esperaron en un aula separada, en la que se cuidó que no hablaran entre ellos. Mientras tanto, a los otros tres participantes (roles A, B y C) se les entregaron cuatro formatos de respuesta para que en cada periodo escribieran su número elegido y sus creencias. Los formatos incluían una clave de identificación única para cada participante, el número de periodo y los espacios para que los participantes escribieran su número elegido y su estimación de los números a elegir por los otros dos participantes. Un ejemplo de estos formatos se puede consultar en el apéndice.\\

Los participantes A, B y C jugaron cuatro periodos de p-beauty contest, durante los cuales registraron tanto su propia elección como sus predicciones acerca de las elecciones de sus oponentes. Al terminar cada periodo, luego de que los participantes llenaran sus formatos, estos se recogían y se anotaban en el pizarrón del aula todos los números elegidos, sin indicar qué participante había elegido cada uno. Inmediatamente y de manera explícita, se calculaban el promedio de estos números y el número objetivo, y se anotaban en el pizarrrón. Una vez hecho esto, se anunciaba qué jugador o jugadores habían elegido el número más cercano al número objetivo, y se les anotaba una ganancia de 6 puntos de juego distribuidos entre todos los ganadores. También se estableció qué jugadores habían estimado acertadamente los números elegidos por los otros jugadores con un margen de error de 5 números de diferencia y se les anotó una ganancia de 1 punto por cada acierto. Una vez realizado este proceso, se devolvía su formato de respuesta a cada candidato, incluyendo en el reverso de los mismos el total de puntos de juego obtenidos en ese periodo (que podía oscilar entre 0 y 8). Luego de que los participantes conocieran los puntos que obtuvieron en cada periodo, se borraba el pizarrón y se daba paso al siguiente periodo.\\

Al terminar los cuatro primeros periodos de juego (El subjuego 1), se contabilizaron los puntos obtenidos por cada participante y se anunciaba al participante ganador de la mayor cantidad de puntos. Si el jugador con más puntos era el participante B o C, se le anunciaba que obtenía otro medio punto extra adicional en su examen, además del medio punto que ya había obtenido por participar en el experimento. Si la persona con más puntos era el participante A, se le decía que para obtener el medio punto extra adicional debía ganar también en el siguiente subjuego.\\

Tras anunciar el ganador del Subjuego 1, se despedía a los participantes B y C, y los participantes D y E ingresaban al aula. Antes de comenzar con el Subjeugo 2, se dio un breve recordatorio de las reglas del juego, haciendo hincapié en que el participante A contaba con más experiencia que los jugadores D y E, al haber jugado cuatro periodos.\\

El subjuego 2 se llevó a cabo de la misma forma que el subjuego 1. Al final de cuatro periodos se contabilizaron los puntos de juego adquiridos por cada participante, y el que obtuvo la mayor cantidad recibió el medio punto extra sobre su examen, además del medio punto extra otorgado por haber participado en el experimento.\\

Con el término del subjuego 2 y el anuncio del participante ganador, se despedía a los participantes y se daba por terminada la sesión.\\
