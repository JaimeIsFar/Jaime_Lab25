% Chapter 1

\chapter{Resultados} % Main chapter title
\label{Cap_Res} % For referencing the chapter elsewhere, use \ref{Cap_Res} 

\section{Análisis Bayesiano}

Para el análisis de los resultados obtenidos, se utilizaron tanto pruebas estadísticas bayesianas como pruebas clásicas o frecuentistas. En el presente apartado se presentan únicamente los resultados de los análisis bayesianos, mientras que los resultados de los análisis clásicos se incluyen como apéndice; en general, ambas aproximaciones apuntan hacia las mismas conclusiones.\\

Se decidió presentar los resultados del análisis bayesiano porque se considera que el factor de Bayes, que funciona como una alternativa al p-value como medida de significancia estadística, refleja de forma mucho más clara la  relación entre los datos obtenidos y las hipótesis nula y alternativa, al señalar cuántas veces es más probable que la evidencia haya sido producida bajo la hipótesis alterna respecto a la hipótesis nula ($BF_{10}$), o viceversa ($BF_{01}$).

\section{Consistencia entre creencias y elecciones en el primer subjuego}

El primer subjuego del presente experimento presenta una replicación parcial del experimento de Lahav \parencite*{Lahav2015}. En ambos casos se juega $p-$beauty contest por varios periodos y se utiliza un método para provocar creencias. Las diferencias radican en el número de jugadores que participan en un periodo (3 en el presente experimento, y hasta 20 en el de Lahav), la cantidad de periodos jugados (4 en el presente experimento y 5 en el de Lahav), y el método utilizado para evocar las creencias (en el presente experimento se preguntó directamente por el número que se creía elegirían los otros jugadores, mientras que en el experimento de Lahav se estimaba el número de jugadores que elegirían un número dentro de ciertos rangos).\\

Para evaluar la consistencia entre las creencias de los participantes y sus números elegidos se emplearon dos métodos: el primero de ellos, computa la diferencia normalizada entre las creencias y las elecciones de los participantes de acuerdo con las elecciones promedio observadas en cada periodo \parencite{Lahav2015}; el segundo, calcula la diferencia relativa entre creencias y elecciones a partir del punto medio entre ambos valores (usado por Slonim, \parencite*{Slonim2005}, para calcular el cambio relativo de los números elegidos por los jugadores de un periodo al siguiente). A su vez, igual que Lahav, para evaluar el cómputo realizado por los participantes, se utilizaron estos métodos de dos formas,  incluyendo u omitiendo la multiplicación por el parámetro $p$.\\

El procedimiento sugerido por Lahav \parencite*{Lahav2015} para calcular las diferencias normalizadas entre las creencias y las elecciones de cada participante en cada periodo, fue incorporado a partir de la siguiente ecuación:\\

\begin{center}
$DN_i^t = \frac{(\frac{2}{3}B_i^t - C_i^t)}{C_t}$\\
\end{center}

Donde $DN_i^t$ es la \textit{Diferencia Normalizada} entre creencias y elecciones de cada participante $i$ en el periodo $t$, computada a partir de la diferencia entre  la media de los números que el participante $i$ estimó que elegirían los otros dos jugadores en el periodo $t$ multiplicado por $\frac{2}{3}$,  ($\frac{2}{3}B_i^t$) y el número elegido por el participante $i$ en el periodo $t$ ($C_i^t$), dividida por el promedio de los números elegidos por todos los participantes en el periodo $t$ ($C_t$).\\

Una vez computadas las diferencias por cada sujeto y periodo, se calculó el promedio de las mismas para poder someterlas a un análisis estadístico que permitiera evaluar si estas fueron significativamente diferentes de $0$. Para ello, se realizaron pruebas-T bayesianas de una sola muestra para cada periodo de juego. En la Tabla~\ref{DN_Sub1} se reporta el factor de Bayes obtenido en dicho análisis, que permite estimar qué tantas veces es más probable que la evidencia corresponda con la hipótesis alterna (hay diferencia entre creencias y elecciones) respecto a la hipótesis nula (no hay diferencia entre creencia y elecciones). Como puede verse, solamente se encontraron diferencias significativas entre creencias y elecciones en los primeros dos periodos del juego.\\

\begin{table}
\caption[Diferencias Normalizadas en el Subjuego 1]{Prueba-T bayesiana de una sola muestra que compara contra 0 las diferencias promedio entre las creencias y las elecciones de los jugadores en cada periodo del primer subjuego.}
\label{DN_Sub1}
\centering
\begin{tabular}{l l | c c c}  %'|' señala Líneas verticales
\toprule
%\tabhead{Groups} & \tabhead{Treatment X} & \tabhead{Treatment Y} \\
\textbf{} & \textbf{$BF_{10}$} & \textbf{Error$\%$} & \textbf{Mean Difference}\\
\midrule
Periodo 1 & 19.300 & 1.823$e^{-6}$ & -0.366 \\
Periodo 2 & 34.545 & 3.137$e^{-4}$ & -0.342 \\
Periodo 3 & 0.281 & 2.840$e^{-5}$ & -0.097 \\
Periodo 4 & 0.652 & 0.015 & -0.147 \\
\bottomrule
\end{tabular}
\end{table}

En la Figura~\ref{fig:DiferenciasNormalizadas_Subjuego1} se presentan las distribuciones prior y posterior computadas en cada periodo. Las distribuciones prior señalan la hipótesis nula (las diferencias estarán cercanas a $0$) y la distribución posterior presenta el estimado realizado a la luz de los datos. La forma más sencilla de interpretar estas gráficas es como una razón de probabilidades: si la densidad de probabilidad es mayor en la distribución prior que en la distribución posterior para el punto que señala un tamaño del efecto $0$ ($\delta = 0$), quiere decir que la evidencia favorece la hipótesis alterna, ya que que a la luz de la evidencia es \textit{muy poco probable} (menos de lo que se esperaba de acuerdo a la distribución prior) que el tamaño de efecto tenga un valor cercano a $0$.

\begin{figure}[th]
\centering
\includegraphics[width=0.70\textwidth]{Figures/Fig_1} 
%\decoRule
\caption[Distribuciones prior y posterior de las Diferencias Normalizadas en el primer Subjuego, (prueba T bayesiana de una muestra)]{Se presenta la comparación entre las distribuciones prior y las distribuciones posteriores computadas en las pruebas-T bayesianas de una sola muestra, por cada uno de los cuatro periodos contenidos en el primer subjuego.}
\label{fig:DiferenciasNormalizadas_Subjuego1}
\end{figure}

Estos resultados son consistentes con lo que reporta Lahav \parencite*{Lahav2015}: en los primeros periodos no hay consistencia entre creencias y elecciones, pero esta parece adquirirse conforme avanzan los periodos. Así mismo, en todos los periodos se encontraron diferencias negativas entre creencias y elecciones, lo que indica que en promedio las creencias de los participantes estuvieron por debajo de sus elecciones reales.\\

En el estudio presentado por Lahav \parencite*{Lahav2015}, el cómputo de la diferencia entre las creencias y las elecciones se repitió omitiendo la multiplicación por $p$, en un intento por evaluar si las inconsistencias halladas entre las creencias y las elecciones se debían a que los participantes no hubieran realizado dicha operación. El presente trabajo también incorporó dicha variación del análisis, que se llevó a cabo de acuerdo a la siguiente ecuación:\\

$DN_i^t = \frac{(B_i^t - C_i^t)}{C_t}$\\

Nuevamente, las diferencias promedio computadas en cada periodo asumiendo que los participantes no multiplicaron sus creencias por $p$, fueron evaluadas con pruebas-T bayesianas de una sola muestra.  Este análisis arrojó resultados inversos a los encontrados cuando la multiplicación por $p$ fue tomada en cuenta: los primeros periodos no muestran diferencias significativas y los últimos, sí.  Aunado a ello, las diferencias en los periodos 3 y 4 se vuelven positivas (indicando que las creencias cayeron por encima de las elecciones). Estos resultados se presentan en la Tabla~\ref{DN_Sub1_noP}.

\begin{table}
\caption[Diferencias Normalizadas en el Subjuego 1 (sin multiplicación por $p$)]{Prueba-T bayesiana de una sola muestra que compara contra 0 las diferencias promedio entre las creencias y las elecciones de los jugadores en cada periodo del primer subjuego.}
\label{DN_Sub1_noP}
\centering
\begin{tabular}{l l | c c c}  %'|' señala Líneas verticales
\toprule
%\tabhead{Groups} & \tabhead{Treatment X} & \tabhead{Treatment Y} \\
\textbf{} & \textbf{$BF_{10}$} & \textbf{Error$\%$} & \textbf{Mean Difference}\\
\midrule
Periodo 1 & 0.207 & 0.010 & -0.049 \\
Periodo 2 & 0.196 & 0.013 & -0.012 \\
Periodo 3 & 3.811 & 3.017$e^{-6}$ & 0.355 \\
Periodo 4 & 1.861 & 4.032$e^{-6}$ & 0.280 \\
\bottomrule
\end{tabular}
\end{table}

De acuerdo con el factor de Bayes, aunque la hipótesis alterna es más probable en los periodos 3 y 4 (es decir, parece ser que sí hay diferencias entre las creencias y las elecciones), la evidencia a favor de esta conclusión es relativamente pequeña, (particularmente en el periodo 4, donde podría considerarse anecdótica). En la Figura~\ref{fig:DiferenciasNormalizadas_Subjuego1_NoP} se incluyen las distribuciones prior y posteriores de cada periodo.

\begin{figure}[th]
\centering
\includegraphics[width=0.70\textwidth]{Figures/Fig_2} 
%\decoRule
\caption[Distribuciones prior y posterior de las Diferencias Normalizadas en el primer Subjuego, (prueba T bayesiana de una muestra), sin incluir la multiplicación por $p$]{Se presenta la comparación entre las distribuciones prior y las distribuciones posteriores computadas en las pruebas-T bayesianas de una sola muestra, de acuerdo con las diferencias normalizadas computadas por participante, en cada uno de los cuatro periodos contenidos en el primer subjuego, entre sus elecciones y creencias, asumiendo que estas no se multiplican por p.}
\label{fig:DiferenciasNormalizadas_Subjuego1_NoP}
\end{figure}

Este resultado difiere considerablemente de los hallazgos reportados por Lahav \parencite*{Lahav2015}, donde al excluir la multiplicación por $\frac{2}{3}$, las diferencias en los cuatro periodos se volvieron positivas, significativamente diferentes de $0$, y en general,  más grandes que cuando sí se consideraba la multiplicación por $\frac{2}{3}$. Estos resultados fueron interpretados como un indicador de que los participantes sí tomaron en cuenta multiplicación por $p$.\\

En el análisis de los datos obtenidos en el presente estudio, cuando se incluye la multiplicación por $p$ en el cálculo de las diferencias entre creencias y elecciones, se encuentra que éstas fueron significativas en los primeros periodo, en tanto que  al excluir dicha multiplicación, las diferencias significativas se reportan sólo en los últimos periodos, aunque la evidencia parece ser débil. En el caso del presente estudio, estos resultados pueden apuntar a que los participantes comienzan el juego sin considerar la multiplicación por $\frac{2}{3}$, pero la incorporan en sus decisiones al avanzar los periodos (o por lo menos, aprenden que el número objetivo siempre está por debajo del número promedio).\\

En general, el presente estudio presenta hallazgos similares a los reportados por Lahav \parencite*{Lahav2015} al emplear el método propuesto para calcular las diferencias normalizadas entre las creencias y las elecciones de los participantes, aún a pesar de las diferencias en el método de provocación de creencias:\\

\begin{itemize}
\item Existen discrepancias entre las creencias y las elecciones de los participantes en los primeros periodos de juego, pero no en los últimos.\\

\item La elección real de los participantes se encuentra entre su percepción de cuál será el número promedio y cuál será el número objetivo.\\
\end{itemize}

Siendo que la diferencia más importante entre lo hallado en el presente estudio y lo reportado por Lahav \parencite*{Lahav2015},  es que, en promedio, los participantes no parecen incorporar la multiplicación por $\frac{2}{3}$ en su elección, al menos en los primeros periodos.\\

Además de replicar el método de diferencias normalizadas utilizado por Lahav \parencite*{Lahav2015}, las diferencias entre creencias y elecciones fueron evaluadas con un segundo método que no dependía de la elección promedio de todos los participantes en cada periodo para ponderarlas. La medida utilizada fue la diferencia relativa entre creencias y elecciones de cada participante $i$ en cada periodo $t$, calculada de la siguiente manera:\\

$DR_i^t= \frac{(\frac{2}{3}B_i^t - C_i^t)}{0.5(\frac{2}{3}B_i^t + C_i^t)}$\\


Las diferencias relativas computadas en promedio por cada periodo fueron evaluadas en términos de qué tanto se alejaron de $0$, mediante la realización de pruebas-T bayesianas de una sola muestra, que se presentan en la Tabla~\ref{DR_Sub1}. Se encontró que en tres de los cuatro periodos hubo diferencias significativas entre creencias y elecciones. En la Figura~\ref{fig:DiferenciasRelativas_Subjuego1} se incluyen las distribuciones prior y posterior de cada periodo.

\begin{table}
\caption[Diferencias Relativas en el Subjuego 1]{bayesiana de una sola muestra que compara contra 0 las diferencias promedio entre las creencias y las elecciones de los jugadores en cada periodo del primer subjuego.}
\label{DR_Sub1}
\centering
\begin{tabular}{l l | c c c}  %'|' señala Líneas verticales
\toprule
%\tabhead{Groups} & \tabhead{Treatment X} & \tabhead{Treatment Y} \\
\textbf{} & \textbf{$BF_{10}$} & \textbf{Error$\%$} & \textbf{Mean Difference}\\
\midrule
Periodo 1 & 77.283 & 7.775$e^{-5}$ & -0.457 \\
Periodo 2 & 12.797 & 2.047$e^{-6}$ & -0.328 \\
Periodo 3 & 0.214 & 0.008 & -0.052 \\
Periodo 4 & 1.871 & 4.022$e^{-6}$ & -0.212 \\
\bottomrule
\end{tabular}
\end{table}

\begin{figure}[th]
\centering
\includegraphics[width=0.70\textwidth]{Figures/Fig_3} 
%\decoRule
\caption[Distribuciones prior y posterior de las Diferencias Relativas en el primer Subjuego, (prueba T bayesiana de una muestra)]{Se presenta la comparación entre las distribuciones prior y las distribuciones posteriores computadas en las pruebas-T bayesianas de una sola muestra, de acuerdo con las diferencias relativas computadas en cada uno de los cuatro periodos contenidos en el primer subjuego, entre sus elecciones y creencias.}
\label{fig:DiferenciasRelativas_Subjuego1}
\end{figure}

Tal y como se encontró con el método de Diferencias Normalizadas, todas las diferencias son negativas, indicando que consistentemente las creencias estuvieron por debajo de las elecciones reales. Sin embargo, contrario a lo que se esperaría con base en los resultados hallados con el método de Diferencias Normalizadas donde los participantes reducen la inconsistencia entre sus elecciones y las creencias evocadas conforme adquieren experiencia, con el método de Diferencias Relativas se observan inconsistencias (diferencias estadísticamente significativas) entre creencias y elecciones en el periodo 4, aunque no en el periodo 3. De cualquier forma, de acuerdo con el factor de Bayes, la evidencia a favor de la hipótesis alterna en el periodo 4 es anecdótica.\\

Posteriormente, se computaron las Diferencias Relativas omitiendo la multiplicación por $p$, de la siguiente forma:\\

$DR_i^t = \frac{(B_i^t - C_i^t)}{0.5(B_i^t + C_i^t)}$\\

Se realizaron pruebas-T de una sola muestra para determinar si las Diferencias Relativas promedio en cada periodo son significativamente diferentes de 0. En la Tabla~\ref{DR_Sub1_NoP} se presentan los resultados obtenidos por la prueba-T bayesiana. A su vez, la Figura~\ref{fig:DiferenciasRelativas_Subjuego1_NoP} presenta la relación entre las distribuciones prior y las posteriores de cada periodo.


\begin{table}
\caption[Diferencias Relativas en el Subjuego 1 (sin multiplicar por $p$)]{Prueba-T bayesiana de una sola muestra que compara contra 0 las diferencias promedio entre las creencias y las elecciones de los jugadores en cada periodo del primer subjuego.}
\label{DR_Sub1_NoP}
\centering
\begin{tabular}{l l | c c c}  %'|' señala Líneas verticales
\toprule
%\tabhead{Groups} & \tabhead{Treatment X} & \tabhead{Treatment Y} \\
\textbf{} & \textbf{$BF_{10}$} & \textbf{Error$\%$} & \textbf{Mean Difference}\\
\midrule
Periodo 1 & 0.269 0& 5.246$e^{-4}$ & -0.101 \\
Periodo 2 & 0.211 & 0.009 & -0.044 \\
Periodo 3 & 8.393 & 2.307$e^{-6}$ & -0.315 \\
Periodo 4 & 0.810 & 5.669$e^{-6}$ & -0.167 \\
\bottomrule
\end{tabular}
\end{table}

\begin{figure}[th]
\centering
\includegraphics[width=0.70\textwidth]{Figures/Fig_4} 
%\decoRule
\caption[Distribuciones prior y posterior de las Diferencias Relativas en el primer Subjuego, sin la multiplicación $\cdot p$ (prueba T bayesiana de una muestra)]{Se presenta la comparación entre las distribuciones prior y las distribuciones posteriores computadas en las pruebas-T bayesianas de una sola muestra, de acuerdo con las diferencias relativas computadas en cada uno de los cuatro periodos contenidos en el primer subjuego, entre sus elecciones y creencias, sin tomar en cuenta la multiplicación por p.}
\label{fig:DiferenciasRelativas_Subjuego1_NoP}
\end{figure}

Similar a lo observado cuando se omitió la multiplicación por p en el método de Diferencias Normalizadas propuesto por Lahav \parencite*{Lahav2015}, se encontró una reversión en la significancia reportada en todos los periodos, aunque nuevamente, la evidencia en el periodo 4 es anecdótica. En tres de los cuatro periodos se observaron diferencias positivas. Esto indica que con el uso del método de Diferencias Relativas, en promedio, las creencias están más cercanas y ligeramente por arriba de las elecciones reales de los participantes, cuando no se toma en cuenta la multiplicación por $p$.\\

Aunque cada uno de los métodos empleados para la evaluación de la consistencia entre las creencias y las elecciones de los participantes compensa la tendencia hacia el equilibrio de forma diferente (y el problema de suelo resultante), los dos mostraron resultados muy similares: en los primeros periodos las diferencias entre creencias y elecciones son grandes, pero se reducen en los periodos posteriores, sugiriendo que los jugadores se vuelven consistentes conforme adquieren experiencia y aprenden también que para acercarse al número objetivo necesitan elegir números por debajo del número promedio.\\

\section{Efecto de Reset}

De acuerdo con los resultados reportados por Slonim \parencite*{Slonim2005}, los jugadores presentan un efecto de “reset” en la tendencia a elegir números cada vez más pequeños cuando los otros jugadores son reemplazados por nuevos jugadores que no tienen experiencia en el juego. Tomando estos hallazgos en cuenta, el presente estudio incorporó un Subjuego 2, donde sólo uno de los jugadores del Subjuego 1 permaneció jugando por otros cuatro periodos mientras el resto fue reemplazado por jugadores nuevos. Esta manipulación se hizo para evaluar la tendencia que en el estudio de Lahav \parencite*{Lahav2015} lleva a asumir que los jugadores se vuelven más consistentes conforme adquieren experiencia. En otras palabras, agregar un segundo Subjuego permite evaluar, con base en las respuestas del participante con experiencia, si la consistencia entre las elecciones y las creencias es algo que se adquiere con la experiencia o si es sólo el resultado del efecto de suelo asociado a la tendencia típicamente reportada en cualquier serie de juegos p-beauty contest repetido a elegir números cada vez más pequeños.\\

Para poder comparar los resultados del subjuego 1 con el subjuego 2, fue necesario determinar si la incorporación de nuevos jugadores al inicio del subjuego 2 interrumpió la tendencia a elegir números cada vez más pequeños en el jugador que se mantuvo en el juego (participante A). Es decir, para corroborar que el diseño experimental propuesto permite responder a la cuestión de si las diferencias entre creencias y elecciones se reducen como reflejo de una consistencia adquirida o como producto del efecto de suelo, es necesario evaluar la presencia del efecto de Reset reportado por Slonim \parencite*{Slonim2005}.\\

En la Figura~\ref{fig:Cambios_NumElegido} se muestran los cambios en las elecciones de los participantes A entre periodos consecutivos. Los primeros tres cuadros muestran los cambios dentro del subjuego 1.  El cuarto cuadro  evalúa directamente el Efecto de Reset al presentar el cambio entre el último periodo del subjuego 1 y el primer periodo del subjuego 2. En estos cuadros, los puntos que caen por debajo de la línea de identidad indican que se eligieron números más pequeños de un periodo a otro, con lo que se observa que más participantes A eligen números más pequeños entre los periodos del primer subjuego (por ejemplo, el $80\%$ reduce su elección entre el periodo 3 y el periodo 4), y que esta tendencia se revierte al iniciar el subjuego 2 (el $70\%$ incrementa su número).\\

\begin{figure}[th]
\centering
\includegraphics[width=0.70\textwidth]{Figures/Fig_5} 
%\decoRule
\caption[Cambio en el número elegido periodo a periodo.]{Cambio en el número elegido periodo a periodo.}
\label{fig:Cambios_NumElegido}
\end{figure}

De acuerdo con una prueba binomial bayesiana de una cola, los datos presentan evidencia anecdótica en favor de la hipótesis alterna que establece que los participantes eligen números más pequeños entre cada periodo del subjuego 1 y números más grandes al iniciar el subjuego 2 (ver Tabla 5). La falta de robustez en la evidencia puede deberse a que solamente se realizaron 10 sesiones experimentales.

\begin{table}
\caption[Prueba Binomial para comprobar la proporción de casos en que ]{Prueba binomial bayesiana de los casos en los que los jugadores eligen un número más grande que en el periodo anterior (El periodo 5 representa el primer periodo del Subjuego 2).}
\label{DR_Sub1_NoP}
\centering
\begin{tabular}{l l | c c c}  %'|' señala Líneas verticales
\toprule
%\tabhead{Groups} & \tabhead{Treatment X} & \tabhead{Treatment Y} \\
\textbf{} & \textbf{$BF_{10}$} & \textbf{Error$\%$} & \textbf{Mean Difference}\\
\midrule
Periodo 1 & 0.269 0& 5.246$e^{-4}$ & -0.101 \\
Periodo 2 & 0.211 & 0.009 & -0.044 \\
Periodo 3 & 8.393 & 2.307$e^{-6}$ & -0.315 \\
Periodo 4 & 0.810 & 5.669$e^{-6}$ & -0.167 \\
\bottomrule
\end{tabular}
\end{table}