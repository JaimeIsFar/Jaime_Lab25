\documentclass[jou,apacite]{apa6}

\usepackage{hanging}

\title{Exploring the consistency in Choices registered in the p-Beauty Contest as a reflection of participants' beliefs}
\shorttitle{APA style}

\twoauthors{J.O. Islas}{A. F. Ch\'{a}vez}
\twoaffiliations{National Autonomous University of Mexico}{National Autonomous University of Mexico}

\abstract{Este paper va a estar bien padriurix. Este paper va a estar bien padriurix. Este paper va a estar bien padriurix. Este paper va a estar bien padriurix.}


\rightheader{APA style}
\leftheader{Author One}

\begin{document}
\maketitle    

\section{Introduction}

Decisions often take place in environments that change over time. Availability of food in foraging animals may vary gradually as a function of source growth or continuous intake, so the position of objects in space moving at a certain velocity. Having accurate beliefs under such circumstances allows behavior to be better allocated and to optimize reward, e.g., by changing to a richer foraging location or predicting the correct position of an object moving towards us. In stable environments, error-driven algorithms approximate the state of the world by reducing the discrepancy of estimates and outcomes as new observations arrive. One common expression to compute this is the Delta rule: $$\delta_t = r_t - \mu_t$$
\begin{equation}
\mu_{t+1} = \mu_t + \alpha\delta
\end{equation}

where, at a given trial t, the prediction error $\delta_t$ , weighted by the learning rate $\alpha$, is used to update the current estimate $\mu_t$ after outcome $r_t$  is observed. Evidence from Experimental Psychology (Bush \& Mosteller, 1951; Rescorla \& Wagner, 1972; Miller et al., 1995) and Neuroscience (Schultz, 1997; Niv, 2013) provides support for this algorithm as a plausible mechanisms of learning in mammals, and it has also been implemented as an effective solution in multiple machine learning problems (Sutton \& Barto, 1988). However, one of its limitations is the inability to predict behavior in non-stationary environments partly due to the fixed nature of the learning rate parameter. For example, in change-point problems, having a low $\alpha$ makes predictions during stable periods accurate but causes a slow adaptation after a change. A high $\alpha$ has the opposite effect, making inaccurate predictions during stability but having a quick adaptation to changes. Adjusting this parameter after the change-point (Nassar et al., 2010) and using multiple delta rules with their own learning rates (Wilson et al., 2013) are some of the possible solutions reported on literature.  

Standard Delta rule model also has difficulties performing in environments that change gradually in time.  Consider the simulation shown in figure 1.  In this example, availability of food on a given day (blue dots) is given by sampling from a Gaussian distribution with variance 1 and mean X changing by $X_{t+1}$ = $X_t$ + $v_t$ ,where $v_t$  is a velocity term following a random walk with variance 1. In 1A, predictions of Delta rule (red line) are plotted using the simulated data and $\alpha$ = 0.1. In this case, previous estimations outweigh new observations, making the algorithm change so slowly that is unable to track the generative mean. Figure 1B shows the same sequence using $\alpha$ = 0.9. Although it does a better job than 1A, given the high learning rate, the Delta rule predictions on a given day resemble the just-perceived outcome the day before. As a result, when the generative mean moves upwards (trial 4 to 12) or downwards (trial 18 to 23) the model adapts as though it were one step behind. 

\begin{figure}[!ht]
\centering
\includegraphics[height=5cm,width=7cm]{SimulationDelta.png}
\caption{Simulated data for availability of food changing over days. (A) Delta rule model with $\alpha$ = 0.2 weights previous estimations more than new outcomes, making it perform poorly. (B) Delta rule model with $\alpha$ = 0.9 responds inaccurately as the mean of observations begins to drift (trial 4-12 and 18 to 23). Its high learning rate makes future predictions roughly the same as just-perceived outcomes.}
\label{fig:hiperpro}
\end{figure}

One way around this problem is to estimate in a single number the rate at which the generative process changes. This quantity can be updated similarly to the delta rule algorithm as new observations arrive and added linearly to the final prediction. Such a model leads to a simple version of a Kalman Filter (Kalman, 1960; Sutton, 1992). Previous work has shown that Kalman Filter is a useful tool when modelling behavior of subjects in drifting environments, for example in Multiple Cue Probability Learning tasks (MCPL) when the cue-outcome relations vary over time (Speekenbrink, 2010), or at the acquisition and extinction of behavioral responses (Kakade \& Dayan, 2000, 2002). Our goal was to build on these studies to model predictions of subjects in a noisy and gradually changing visual task. We implemented a Delta rule model that assumes participants a) estimate the rate of change in the generative process, b) modulate the impact of new observations via their learning rates depending on the level of noise and c) accompany each estimation with a certain degree of uncertainty. 

We developed a novel task where participants have to predict the future location of a spaceship that orbits around the earth. Its position is sampled from a Gaussian distribution with a mean that changes over trials. We set four different values of variance that defined our conditions. 

Our major findings suggest that predictions of participants change gradually in a trial-by-trial manner following the generative process and that a model that assume they estimate the rate of change is efficient in describing their performance. 


\section{Method}

\subsection{Participants}
Hyperboloid model

Tradeoff model 





\section{Results}

Two models were evaluated: one from the alternative-based choice family (Hyperboloid model) and the other from the attributed-based choice family (Trade-off model). The latter model can account for intransitive patters, which the alternative-based models cannot accommodate. The models' comparison was done with Bayesian Modeling in order to infer individual parameters. 

The used notation for describing the bayesian analysis was adopted from Lee \& Wagenmakers (2014). In this representation, shaded nodes correspond to observed variables, whereas unshaded nodes stand for latent variables. Double-bordered nodes are deterministic, while single-bordered nodes are stochastic. Circles represent continuous variables, and squares portray discrete variables. 

\subsection{Discusion}

The evaluated models were analysed for time and probability, therefore there was a total of four bayesian models. The analysis was based on the statistical work of Scholten et al. (2014) were they made model comparison with bayesian analysis. However in the present work, the parametrical estimation was done at an individual level. 
\begin{equation}
Q(w(ts),w(tl))= \frac{\kappa}{\alpha} log\left(1+\alpha \left(\frac{w(tl)-w(ts)}{\vartheta}\right) ^\vartheta \right)
\end{equation}

The four models share the following structure: All of them have three rectangles that enclose independent replications of 1) number of participants i; 2) number of questions j; 3) number of repetitions for the same questions r. The individual responses, $C_ijr$, were modeled with a Bernoulli process with parameter , that represents the probability of choosing the larger reward. The node $x^s_ij$ is the small outcome, and $x^l_ij$ is the larger outcome. 

The difference in between models of time and probability are the following: For the time models, the nodes $t^s_ij$ y $t^l_ij$ are the sooner delay and the later delay, respectively. For the probability task, the probabilities were transformated into odds against the receipt of a reward: $\Omega = \frac{p-1}{p}$. Therefore, the node $\Omega^s_ij$ are the odds against receipt of the safer smaller reward, and $\Omega^l_ij$ are the odds against the receipt of the riskier larger reward. 

Figures tal and tal, are the bayesian graphical models for time and probability of the hyperboloid function. Both have the same structure where the probability of choosing the larger reward, $\theta_ij$ is obtained with the discounted values of both alternatives; where the discounted value of the larger later/riskier is divided between the sum of the discounted value of the larger later/riskier and the discounted value of the smaller sooner/safer. Each discounted value is gathered with the discount factor and the absolute outcome. 

The discount factor $d_ij^sl$ has the hyperboloid structure with two free parameters: the parameter $\kappa_i$ which reflects the degree of discounting (with higher values of it associated with steeper discounting); and the parameter $\tau_i$, which determines the shape of the discounting function (Green, Cita). These parameters were given a lognormal distribution with a mean of zero and a standard deviation of one. This distribution does not allow negative values, it is heavy tailed, and it puts little weight on values extremely close to zero. 

\begin{figure}[!ht]
\centering
\includegraphics[height=5cm,width=7cm]{ejemplopregu.pdf}
\caption{hola}
\label{fig:hiperpro}
\end{figure}


\section{References}

\begin{hangparas}{.3in}{1}

Gallistel, C. R., Krishan, M., Liu, Y., Miller, R., \& Latham, P. E. (2014). The perception of probability. {\it Psychological review, 121(1),} 96.

Kalman, R. E. (1960). A new approach to linear filtering and prediction problems. {\it Journal of basic Engineering,} 82(1), 35-45.

Miller R.R., Barnet R.C. \& Grahame NJ (1995) Assessment of the Rescolra-Wagner model. {\it Psychological Bulletin} 117: 363–386.

Nassar, M. R., Wilson, R. C., Heasly, B., \& Gold, J. I. (2010). An approximately Bayesian delta-rule model explains the dynamics of belief updating in a changing environment. {\it Journal of Neuroscience,} 30(37), 12366-12378.

Rescorla R.A. \& Wagner A.R. (1972) A Theory of Pavlovian conditioning: variations in the effectiveness of reinforcement and nonreinforcement. In: Black AH, Prokasy WF, editors, Classical conditioning II: current research and theory. New York: Appleton Century Crofts. chapter 3. pp. 64–99. 

Ricci, M. \& Gallistel, R. (2017). Accurate step-hold tracking of smoothly varying periodic and aperiodic probability.{\it Attention, Perception, \& Psychophysics, }1-15.

Schultz, W., Dayan, P., \& Montague, P. R. (1997). A neural substrate of prediction and reward. {\it Science,} 275(5306), 1593-1599.

Speekenbrink, M., \& Shanks, D. R. (2010). Learning in a changing environment. {\it Journal of Experimental Psychology: General,} 139(2), 266.

Sutton, R. S. (1992). Gain adaptation beats least squares. In {\it Proceedings of the 7th Yale workshop on adaptive and learning systems} (Vol. 161168).

Sutton, R. S., \& Barto, A. G. (1998). {\it Introduction to reinforcement learning (Vol. 135)} . Cambridge: MIT Press.

Wilson, R. C., Nassar, M. R., \& Gold, J. I. (2013). A mixture of delta-rules approximation to Bayesian inference in change-point problems. {\it PLoS computational biology,} 9(7), e1003150.


\end{hangparas}


\onecolumn

\bibliography{sample}

\end{document}