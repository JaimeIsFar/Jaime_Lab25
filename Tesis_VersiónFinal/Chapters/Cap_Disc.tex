% Chapter Template

\chapter{Discusión} % Main chapter title

\label{Cap_Disc} % Change X to a consecutive number; for referencing this chapter elsewhere, use \ref{ChapterX}

El experimento realizado aporta evidencia acerca de la relación que existe entre la elección de un número en el juego de $p$-beauty contest y las creencias que se reportan sobre las elecciones de los demás jugadores, mediante la incorporación de una variación del método para explicitar creencias propuesto por Lahav (\citeyear{Lahav}), en un procedimiento que aprovecha el efecto de Reset reportado por Slonim (\citeyear{Slonim}), para evaluar el posible impacto que tiene la experiencia sobre la consistencia con que las elecciones de los jugadores reflejan sus creencias sobre los demás participantes en el juego. \\

Para evaluar la diferencia entre las elecciones de los jugadores y sus creencias sobre las tiradas de sus oponentes, se utilizaron dos métodos diferentes. Primero, se tomó en cuenta la Diferencia Normalizada entre las creencias y las elecciones al ponderar esta por el promedio de los números elegidos por todos los jugadores en cada periodo; y después, se computó la Diferencia Relativa que toma como factor de ponderación el valor intermedio entre las creencias y las elecciones. Ambos métodos buscan compensar la tendencia que presentan las elecciones de los jugadores a converger en un equilibrio cercano a 0 cuando el juego se repite a lo largo de varios periodos. La diferencia sustancial entre ambos, es que la Diferencia Normalizada depende de la elección promedio registrada por todos los jugadores en el periodo a evaluar y la Diferencia Relativa se calcula únicamente a partir de la creencia y elección del jugador en cuestión.\\

Si bien estas dos métodos llevaron al cálculo de valores distintos por cada jugador en cada periodo,  la relación entre estos se presentó de la misma forma: Los jugadores presentan inconsistencias entre sus creencias y sus elecciones cuando no tienen experiencia, tal y como se observó en los primeros periodos jugados por los participantes sin experiencia en el Subjuego 1 (participantes A, B y C) y en  el Subjuego 2 (participantes D y E). Dichas inconsistencias se reducen conforme los participantes adquieren experiencia, hacia el final de cada subjuego, y se mantienen a lo largo de los cuatro periodos que conforman el Subjuego 2 para los participantes A, que juegan con los participantes sin experiencia D y E.\\

En promedio, las elecciones reales de los jugadores se situaron por encima del número objetivo computado a partir de las creencias registradas en cada periodo, y en cambio, se mantuvieron por debajo del promedio de sus creencias.\\

Para evaluar la posibilidad de que las inconsistencias observadas se debieran a que los jugadores no estuvieran tomando en cuenta que el promedio de sus creencias debía multiplicarse por $p$ al momento de elegir su número, se incluyeron variaciones en el cálculo de las Diferencias Normalizadas y Relativas que omitían la multiplicación por $p$. Con ello se observó que la elección de los participantes era más consistente con el promedio de sus creencias (sin incluir la multiplicación por $p$) en los primeros periodos, pero conforme adquirieron experiencia entre periodos, sus elecciones se fueron acercando más a la del número objetivo estimado de acuerdo a sus creencias multiplicadas por $p$ en los últimos periodos del Subjuego 1 y a lo largo de todo el Subjuego 2, para los participantes A. Este resultado indica que los participantes aprenden a incluir la multiplicación por $p$ conforme adquieren experiencia en el juego. De cualquier forma, no es posible determinar si los participantes incorporan el cálculo explícitamente, o simplemente aprenden de forma intuitiva a elegir números cada vez más pequeños, por debajo del promedio de sus creencias.\\

El inicio del Subjuego 2 estuvo marcado por la introducción de dos nuevos jugadores (D y E) que reemplazaron a dos de los participantes en el Subjuego 1 (B y C), siendo que uno de los jugadores de dicho Subjuego permaneció durante cuatro periodos más. Con esta manipulación experimental se replicó exitosamente el efecto de Reset reportado por Slonim (\citeyear{Slonim}), permitiendo evaluar la consistencia entre las elecciones y las creencias de los participantes A como una función de su experiencia, sin la influencia del efecto de suelo. Los resultados obtenidos a este respecto confirman la importancia que tiene la experiencia de los participantes sobre su desempeño en el juego de $p$-beauty contest, en términos de la consistencia entre los números elegidos y los números que predijeron que tirarían los demás jugadores. \\

El experimento que se realizó para conducir este estudio se llevó a cabo en 10 sesiones experimentales compuestas de 8 periodos de juego. En cada periodo 3 jugadores elegían números e intentaban adivinar los números de los otros jugadores. Considerando la cantidad de datos generados con este diseño y los diseños utilizados en la literatura (Kocher et al., \citeyear{Kocher}, Slonim, \citeyear{Slonim}), se considera que 10 sesiones fueron suficientes para obtener resultados válidos, pero que replicar el experimento con una muestra más grande podría incrementar la robustez de los hallazgos reportados.\\

