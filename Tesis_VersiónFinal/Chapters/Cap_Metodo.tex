% Chapter Template

\chapter{Método} % Main chapter title

\label{Cap_Exp} % Change X to a consecutive number; for referencing this chapter elsewhere, use \ref{ChapterX}

%----------------------------------------------------------------------------------------
%	SECTION 1
%----------------------------------------------------------------------------------------
\section{Participantes}

Se contó con la participación de 50 estudiantes de los primeros semestres de la carrera de Psicología en la Facultad de Psicología de la Universidad Nacional Autónoma de México. A cambio de su participación, todos los estudiantes recibieron medio punto extra sobre la  calificación de uno de sus exámenes, y  los participantes que obtuvieron mayores ganancias al final de cada Subjuego recibieron otro medio punto extra. Su participación en el experimento fue voluntaria, podían abandonarlo si lo deseaban, y se mantuvo el anonimato entre los jugadores durante el experimento y en el análisis de datos, asignando una clave de identificación única a cada participante.\\

\section{Diseño Experimental}

Se realizaron 10 sesiones experimentales con  5 participantes diferentes en cada una. El experimento se llevó a cabo en un aula sin distracciones externas. Al inicio de cada sesión, se asignó aleatoriamente un rol a cada participante (identificado con las letras A, B, C, D y E).\\

Las sesiones experimentales estuvieron compuestas por ocho periodos de p-beauty contest, repartidos en dos Subjuegos de cuatro periodos. En cada Subjuego participaron solo tres personas: En el Subjuego 1, los participantes A, B y C jugaron por cuatro periodos, mientras los participantes D y E esperaban en un aula diferente sin poder hablar entre ellos. Terminado el Subjuego 1, los participantes B y C se retiraban y eran reemplazados para el Subjuego 2 por los participantes D y E. En el Subjuego 2, los participantes A, D y E jugaron por cuatro periodos, al final de los cuales se dio por terminado el experimento.\\

Además de elegir su número en el juego, se solicitó a cada participante que en cada periodo escribieran dos números que creyeran que estarían lo más cerca posible de los números elegidos por los otros dos jugadores, dentro de un márgen de $\frac{+}{-}5$.\\

Se decidió utilizar únicamente dos subjuegos ya que los efectos de la experiencia reportados por Slonim (\citeyear{Slonim}) no mostraron ser significativamente diferentes al agregar un tercer subjuego.\\

Es importante mencionar que el Subjuego 1 es una replicación parcial del experimento de Lahav (\citeyear{Lahav}), dado que es un juego repetido con explicitación de creencias.\\

\section{Procedimiento}

Para cada sesión, se citó a los 5 participantes en un aula donde se les leyeron las instrucciones del experimento: se les explicó cómo jugar p-beauty contest, se les dio la instrucción adicional de intentar adivinar los números a elegir por los otros jugadores (registrando dos creencias por periodo) y se les explicó que la sesión se dividiría en dos Subjuegos con tres participantes en cada uno. Finalmente, se hizo de su conocimiento las ganancias que podrían acumular en caso de dar el número que más se acercara al número objetivo computado a partir de todas los números emitidos (6 puntos para el jugador ganador, que en caso de empate se dividirían equitativamente entre los ganadores), o bien, de acertar en la predicción de las tiradas de sus compañeros (1 punto por cada número que estuviera dentro de un rango de $\frac{+}{-}$ 5 del número elegido por otro jugador). Finalmente, se pidió a los participantes que repitieran las reglas del juego para garantizar que habían entendido las instrucciones (las cuales se pueden consultar en los Apéndices).\\

Los participantes asignados mediante sorteo a los roles D y E esperaron en un aula separada, en la que se cuidó que no hablaran entre ellos. Mientras tanto, a los otros tres participantes (roles A, B y C) se les entregaron cuatro formatos de respuesta para que en cada periodo escribieran su número elegido y sus creencias sobre los números a elegir por sus oponentes. Los formatos incluían una clave de identificación única para cada participante, el número de periodo y los espacios para que los participantes escribieran su número elegido y su estimación de los números a elegir por los otros dos participantes. Un ejemplo de estos formatos se puede consultar en los Apéndices.\\

Los participantes A, B y C jugaron cuatro periodos de p-beauty contest, durante los cuales registraron tanto su propia elección como sus predicciones acerca de las elecciones de sus oponentes. Luego de que los participantes llenaran sus formatos en cada periodo, estos se recogían y se anotaba en el pizarrón del aula todos los números elegidos, sin indicar qué participante había elegido cada uno. Inmediatamente, y de manera explícita, se calculaba el promedio de estos números y el número objetivo, que también se anotaban en el pizarrón. Una vez hecho esto, se anunciaba qué jugador o jugadores habían elegido el número más cercano al número objetivo (con la clave de identificación asignada), y se les anotaba una ganancia de 6 puntos de juego distribuidos entre todos los ganadores. También se estableció qué jugadores habían estimado acertadamente los números elegidos por los otros jugadores con un margen de error de 5 números de diferencia y se les anotó una ganancia de 1 punto por cada acierto. Una vez realizado este proceso, se devolvía su formato de respuesta a cada candidato, indicando en el reverso de los mismos el total de puntos de juego obtenidos en ese periodo (que podía oscilar entre 0 y 8). Luego de que los participantes conocieran los puntos que obtuvieron en cada periodo, se borraba el pizarrón y se daba paso al siguiente periodo.\\

Al terminar los cuatro primeros periodos del Subjuego 1, se contabilizaron los puntos obtenidos por cada participante y se anunciaba al participante ganador. Si el jugador con más puntos era el participante B o C, se le anunciaba que obtendría otro medio punto extra sobre la calificación de su examen, además del medio punto que ya había obtenido por participar en el experimento. Si la persona con más puntos era el participante A, se le decía que para obtener el medio punto extra adicional, debía ganar también en el siguiente Subjuego.\\

Tras anunciar el ganador del Subjuego 1, se despedía a los participantes B y C, y los participantes D y E ingresaban al aula. Antes de comenzar con el Subjuego 2, se dio un breve recordatorio de las reglas del juego, haciendo hincapié en que el participante A contaba con más experiencia, al haber jugado los cuatro periodos anteriores.\\

El Subjuego 2 se llevó a cabo de la misma forma que el Subjuego 1. Al final de cuatro periodos se contabilizaron los puntos de juego conseguidos por cada participante, y el que obtuvo la mayor cantidad recibió el segundo medio punto extra sobre su examen.\\

Con el término del subjuego 2 y el anuncio del participante ganador, se despedía a los participantes y se daba por terminada la sesión.\\

