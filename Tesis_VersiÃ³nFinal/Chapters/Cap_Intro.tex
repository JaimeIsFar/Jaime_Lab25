% Chapter 1

\chapter{Introducción} % Main chapter title

\label{Chapter1} % For referencing the chapter elsewhere, use \ref{Chapter1} 

%----------------------------------------------------------------------------------------

% Define some commands to keep the formatting separated from the content 
\newcommand{\keyword}[1]{\textbf{#1}}
\newcommand{\tabhead}[1]{\textbf{#1}}
\newcommand{\code}[1]{\texttt{#1}}
\newcommand{\file}[1]{\texttt{\bfseries#1}}
\newcommand{\option}[1]{\texttt{\itshape#1}}

%----------------------------------------------------------------------------------------

Teoría de Juegos es la rama de las matemáticas que estudia la toma de decisiones en situaciones de interacción. Dentro de Teoría de Juegos, se estudian situaciones como la que representa el juego del ciempiés (estudiada por primera vez por Rosenthal, en \citeyear{Rosenthal}). Se trata de un juego donde dos jugadores, A y B, toman turnos para elegir si quedarse con una ganancia acumulada o pasar la ganancia al otro jugador. Si algún jugador elige quedarse con la ganancia acumulada, el otro jugador no recibe nada y el juego termina. En cambio, si el jugador elige pasar, el tamaño de la ganancia acumulada aumenta. El juego dura 10 turnos, y la ganancia acumulada aumenta su valor en cada uno. Si en el último turno, que le corresponde al jugador B, este elige pasar, el jugador A se queda con todas las ganancias.\\

De acuerdo con las reglas del juego, si los jugadores quisieran maximizar sus ganancias, deberían elegir ‘pasar’ para que la ganancia acumulada siga aumentando, pero al mismo tiempo, dado que ambos jugadores quieren quedarse con la ganancia, deben elegir quedarse con ella antes de que lo haga el otro jugador. ¿En qué turno del juego sería una buena elección (para cada jugador) quedarse con la ganancia acumulada?\\

El juego establece que en el último turno, cuando la ganancia acumulada tiene su valor más alto, le corresponde elegir al jugador B. Si este jugador elige pasar, todas las ganancias serán para el jugador A, mientras que si decide quedarse con la ganancia acumulada, B obtendrá la mayor ganancia posible. Por lo tanto, el jugador B debería elegir quedarse con la ganancia en el turno 10. Sin embargo, este razonamiento no es un secreto para el jugador A. Si el jugador A sabe que al jugador B le conviene quedarse con la ganancia en el turno 10 (en cuyo caso A se quedaría sin nada), entonces la mejor estrategia para el jugador A consiste en quedarse con la ganancia acumulada en el turno 9. Pero así como el jugador A es capaz de anticipar que el jugador B elegirá quedarse con la ganancia en el turno 10, el jugador B puede deducir que el jugador A elegirá quedarse con la ganancia en el turno 9. Si este es el caso, el jugador B debería elegir quedarse con la ganancia en el turno 8. Si ambos jugadores repiten este razonamiento para todos los turnos de juego, eventualmente se llega a la conclusión de que la mejor estrategia para cualquiera de los jugadores en cualquier turno es elegir quedarse con la ganancia, sin importar el turno en que se encuentren. A este proceso de repetir un razonamiento para llegar a la mejor estrategia disponible tomando en cuenta que el otro jugador también puede hacerlo, se conoce como razonamiento iterado, y es uno de los conceptos básicos en Teoría de Juegos.\\

A las situaciones donde los jugadores eligen la estrategia que maximice su ganancia esperada, tomando en cuenta las estrategias de los otros jugadores, se les conoce como el equilibrio de Nash (Nash, \citeyear{Nash}), otro concepto básico en Teoría de Juegos. En el juego del ciempiés, el equilibrio de Nash supondría que el jugador A se quede con la ganancia acumulada desde el primer turno del juego, incluso si en este punto su valor es muy pequeño.\\

En muchas situaciones de interacción, el razonamiento requerido para llegar al equilibrio es demasiado complejo para ser plausible en términos conductuales, por lo que cuando personas reales se encuentran en dicho tipo de situaciones, sus elecciones suelen ser distintas de lo que supondría el equilibrio. Sin embargo, existe evidencia de que el aprendizaje (producto de jugar un juego de forma repetida) genera una tendencia a converger al equilibrio, de manera que sólo en situaciones donde no se permite que haya aprendizaje se puede atribuir el equilibrio al pensamiento estratégico por sí sólo, (Crawford, Costa-Gomes & Iriberri, \citeyear{Crawford}).\\

El equilibrio se ve reflejado tanto en las elecciones de los participantes del juego, como en sus creencias sobre los otros participantes; jugadores que son racionales (en términos de teoría de decisión) tienen creencias correctas sobre los otros jugadores, si estos también son racionales, (Crawford, Costa-Gomes & Iriberri, \citeyear{Crawford}).\\

Durante el proceso de razonamiento iterado, los jugadores incorporan las creencias que tienen sobre la conducta de los otros jugadores en su toma de decisiones. Keynes (\citeyear{Keynes}), ilustró el proceso de razonamiento iterado con una analogía que se conoce como \textit{Beauty contest}: Un concurso en el que los participantes deben elegir de entre cien fotografías de rostros, cuáles piensan que los demás participantes considerarán que son los más atractivos. Tomando en cuenta que todos los participantes se enfrentan al mismo problema, para ganar no basta con elegir solamente aquellos rostros que piensen que son los más atractivos, o cuáles piensan que los demás participantes piensan que son más atractivos, sino aquellos que piensen que los demás participantes pensarán que los demás participantes piensan que son los más atractivos. Esto implica tres pasos de razonamiento iterado.\\

Un agente totalmente racional debería realizar tantos pasos de razonamiento iterado como fueran necesarios para llegar a la solución por dominancia del juego (el equilibrio de Nash). En la realidad, las personas no se comportan de forma perfectamente racional, y la cantidad de pasos de razonamiento iterado que realizan es limitada, (Stahl & Wilson, \citeyear{Stahl}, Ho, Caremer & Weigelt, \citeyear{Ho}).\\

Experimentalmente, el juego \textit{p-Beauty contest} (llamado así a partir de la analogía de Keynes), ha sido utilizado para estudiar el razonamiento iterado. En este juego participan varios participantes (por lo menos 3) y cada uno debe elegir un número entero dentro el rango $[0 - 100]$, de manera simultánea y sin revelarlo a los otros jugadores. Posteriormente, se calcula la media de todos los números elegidos y este valor se multiplica por un parámetro $p$ que es un número positivo y diferente de 1, conocido de antemano por todos los jugadores, (generalmente se utiliza $p = \frac{2}{3}$). Al valor resultante de este cálculo se le llama el número objetivo, y el ganador del juego será el participante que haya elegido el número más cercano a este número.\\

Si los jugadores fueran perfectamente racionales, creyeran que los demás jugadores también lo son y además pudieran realizar una cantidad infinita de pasos de razonamiento iterado, llegarían a la solución por dominancia del juego mediante el siguiente razonamiento: El valor más grande que puede alcanzar el número objetivo (con $p = \frac{2}{3}$) es $100 \cdot p$, así que cualquier número arriba de este valor es dominado por $66.66$. Jugadores racionales obedecerán la dominancia y creerán que los demás jugadores lo harán también, por lo que todos elegirán un número menor a $66.66$. Por lo tanto, el valor máximo del número objetivo será $100 \cdot p^2$; (siendo $p^2=44.44)$ y elegir cualquier número por arriba de este será una estrategia dominada. Aplicando este razonamiento una y otra vez (de ahí el nombre de razonamiento iterado), el valor del número objetivo se reduce con cada iteración, hasta que se llega al equilibrio de Nash del juego, según el cual todos los jugadores deberían elegir $0$, (Nagel, \citeyear{Nagel}; Ho, Camerer, Weigelt, \citeyear{Ho}).\\

Empíricamente, esto no suele ocurrir, pero cuando el mismo grupo de participantes juegan repetidamente (más de un periodo), se ha reportado consistentemente que sus elecciones se acercan paulatinamente al equilibrio (elegir $0$) con cada periodo, (Nagel, \citeyear{Nagel}, Ho, Camerer & Weigelt, \citeyear{Ho}). También se ha observado que dicha tendencia se interrumpe cuando se agregan nuevos participantes al juego, siendo que los jugadores con experiencia incrementan el número elegido al enfrentarse a estos jugadores novatos, lo que se conoce como Efecto de Reset (Slonim, \citeyear{Slonim}).\\

Se han propuesto varios modelos para dar cuenta de la forma en la que las personas eligen sus números en el juego. Estos modelos capturan la noción de que el número elegido por las personas es un reflejo del número de pasos de razonamiento iterado que son capaces de realizar, (que en la literatura de juegos se conoce como “nivel cognitivo”), así como de las creencias o expectativas que los participantes tienen sobre el nivel cognitivo de los demás jugadores, (Crawford, Costa-Gomes & Iriberri, \citeyear{Crawford}).\\

Algunos estudios, (Agranov et al., \citeyear{Agranov} y Slonim, \citeyear{Slonim}),  han explorado el efecto que tienen las creencias acerca del desempeño de los otros jugadores en las elecciones de cada participante. Cuando esta relación es evaluda de forma indirecta, se ha encontrando evidencia a favor de una relación positiva. En contraste, en estudios que han intentado un acercamiento más directo, se han encontrado inconsistencias entre las creencias reportadas por los participantes sobre lo que harán los otros jugadores y las elecciones registradas; Lahav (\citeyear{Lahav}), utilizó un método en el que le pidió a los jugadores hacer una estimación aproximada de las elecciones de los otros jugadores y comparó directamente estas creencias con sus elecciones reales, encontrando una falta de consistencia entre ambas.\\

El presente trabajo de investigación pretende estudiar de manera directa la relación entre las elecciones de las personas y sus creencias sobre las elecciones de los demás, explorando además si la experiencia que se adquiere en juegos repetidos influye en la relación de estas dos variables.\\

El diseño experimental consiste en juegos repetidos de $p$-beauty contest, estructurado en 2 subjuegos compuestos por 4 periodos (\textit{i.e. repeticiones del juego}) cada uno, por sesión experimental. En cada sesión, sólo un jugador participó en el juego durante los dos subjuegos, siendo que al término del Subjuego 1 los demás jugadores fueron reemplazados por nuevos participantes que no habían jugado previamente. Además de registrar su número elegido en cada periodo, se les solicitó a los jugadores que reportaran directamente los números que creían que los demás jugadores elegirían. Este método de explicitación de creencias permite comparar directamente las elecciones y las creencias registradas por cada jugador, de una forma más evidente a la utilizada por Lahav, (\citeyear{Lahav}). Por otro lado, al igual que en el estudio conducido por Lahav, acercarse en sus predicciones a los números elegidos por los otros jugadores en cada periodo fue recompensado con ganancias en el juego, para agregar motivación a la expresión de las creencias.\\

Se encontró que la entrada de nuevos jugadores incrementó el número elegido por los jugadores con experiencia en el primer periodo del Subjuego 2, siendo que la diferencia entre las creencias y elecciones reportadas en este mismo periodo no incrementa junto con esta. Este hallazgo aporta evidencia de que la adquisición de experiencia incrementa la consistencia.\\

El resto de la presente tesis está dividida en cinco apartados: En el primero, se presenta el marco teórico que describe el modelo de nivel-k usado para explicar la conducta de las personas en este juego, revisa la relación empírica entre las elecciones de las personas y sus creencias y presenta el efecto que ha demostrado tener la experiencia en juegos repetidos sobre las elecciones de los jugadores. En esta misma sección se detallan los objetivos concretos del trabajo de investigación y las estrategias para alcanzarlos. En el segundo apartado se describe el método utilizado, incluyendo información sobre los participantes, el procedimiento y el diseño experimental. En el tercer apartado se presentan los resultados del experimento, se reporta el grado de consistencia que existe entre las creencias y elecciones de los jugadores en el Subjuego 1, el efecto de introducir a participantes sin experiencia en el Subjuego 2 y las diferencias en consistencia entre creencias y elecciones que hay entre los dos subjuegos. Por último, en el cuarto apartado se elabora la discusión a partir de los resultados y las conclusiones se presentan en el quinto apartado.\\
