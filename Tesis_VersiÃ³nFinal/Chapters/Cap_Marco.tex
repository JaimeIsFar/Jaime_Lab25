% Chapter 1
\chapter{Marco Teórico} % Main chapter title

\label{Cap_SDT} % For referencing the chapter elsewhere, use \ref{Chapter1} 

%----------------------------------------------------------------------------------------

% Define some commands to keep the formatting separated from the content 
\newcommand{\keyword}[1]{\textbf{#1}}
\newcommand{\tabhead}[1]{\textbf{#1}}
\newcommand{\code}[1]{\texttt{#1}}
\newcommand{\file}[1]{\texttt{\bfseries#1}}
\newcommand{\option}[1]{\texttt{\itshape#1}}

%----------------------------------------------------------------------------------------

\section{Modelo de nivel-k}

Este modelo fue propuesto por Nagel (\citeyear{Nagel}), para dar cuenta de la conducta de las personas en juegos con solución por dominancia, como es el caso de \textit{p-beauty contest}. El modelo define niveles cognitivos que describen el número de pasos de razonamiento iterado que realiza una persona en el juego y difiere de los modelos de equilibrio clásicos en que las creencias que se tienen sobre las elecciones de los otros jugadores no están basadas en la definición del equilibrio, (Nash, \citeyear{Nash}, Crawford, Costa-Gomes & Iriberri, \citeyear{Crawford}) y por el contrario, admite la posibilidad de que no se considere al resto de los participantes como jugadores perfectamente racionales.\\ 

Según este modelo, los jugadores con un nivel cognitivo 0 serían aquellos que no realizan ningún paso de razonamiento iterado, es decir, que no toman en consideración que las elecciones de los otros participantes influyen en el cálculo del número objetivo. Estos jugadores eligen un número con base en alguna regla arbitraria, (por ejemplo, su número favorito o de la suerte), por lo que podrían elegir cualquier número dentro del rango establecido con una probabilidad similar.\\

Un jugador de nivel 1 es aquel que sí considera que las elecciones de los otros jugadores influyen en el cálculo del número objetivo, pero supone que los otros jugadores no han tomado esto en consideración. El jugador de nivel 1 asume que los demás jugadores son de nivel 0 y elige el número que es la respuesta óptima contra este tipo de jugadores, asumiendo que la media de los números elegidos por todos los jugadores estará cerca de 50 (el mejor predictor de la media de un conjunto de números aleatorios en el rango [0 - 100]) y multiplicando este número por $p$ para acercarse lo más posible al número objetivo.\\

Por su parte, un jugador de nivel 2 no sólo considera que las elecciones de los otros jugadores influyen en el número objetivo, sino que también asume que los otros jugadores saben esto. El jugador de nivel 2 elegirá el número que es la respuesta óptima contra una población de oponentes de nivel 1, y como éstos eligen números cercanos a $50 \cdot p$, el jugador de nivel 2 multiplicará por $p$ nuevamente para acercarse al que piensa que será el número objetivo, esto es $50 \cdot p^2$.\\

En general, un jugador de nivel $k$ elegirá la respuesta óptima contra una población de jugadores de nivel $k-1$, esto es $50 \cdot p^k$.  Con base en esta regla, el modelo computa el nivel cognitivo de los jugadores en función de a cuál de los intervalos de elección establecidos por el modelo pertenece su número elegido.\\

Una variación más sofisticada del modelo de nivel-k es el modelo de jerarquía cognitiva, propuesto por Camerer, Ho, y Chong, (\citeyear{Camerer}). El modelo propone que un jugador de nivel $k$ no sólo da la mejor respuesta contra jugadores $k-1$, sino a una combinación de todos los tipos de jugador desde el nivel 0 hasta $k-1$, a partir de una distribución Poisson que se actualiza de forma bayesiana. El modelo mantiene el supuesto de que las personas consideran que su nivel está por arriba de los demás jugadores, y predice un aprendizaje sobre los niveles $k$ de los otros jugadores.\\

En general, de acuerdo con la evidencia obtenida experimentalmente, los modelos de nivel-k explican mejor el comportamiento de las personas en juegos con solución por razonamiento iterado que los modelos de equilibrio, (Crawford, Costa-Gomes & Iriberri, \citeyear{Crawford}).\\

En los dos modelos previamente descritos, la elección de los jugadores depende de tres elementos: 1) sus creencias sobre cómo juegan los participantes de nivel 0; 2) sus expectativas sobre el nivel cognitivo de sus oponentes; y 3) el número de pasos de razonamiento iterados que son capaces de hacer en el juego, (Agranov et al., \citeyear{Agranov}).\\

En la siguiente sección se ahonda sobre el segundo elemento: las expectativas (\textit{i. e. creencias}) sobre el nivel cognitivo de sus oponentes, y la evidencia que se ha encontrado sobre su relación con los números elegidos.

\section{Relación entre creencias y elecciones}

Para aportar evidencia empírica de la influencia de las creencias que se tiene acerca de la sofisticación de los otros jugadores sobre las elecciones de cada jugador en juegos p-beauty contest, Agranov et al. (\citeyear{Agranov}), manipularon las creencias que los participantes tenían sobre sus oponentes, informando a cada participante que jugaría contra siete estudiantes graduados en Economía con conocimiento sobre este tipo de juegos, o bien, contra siete computadoras programadas para seleccionar aleatoriamente, con la misma probabilidad,  números dentro del rango [0 – 100]. En este estudio se encontró que los números registrados por los participantes cuando se les decía que se enfrentarían a estudiantes graduados corresponden con niveles cognitivos significativamente mayores que cuando creían competir contra las computadoras. Este resultado parece sugerir que el nivel cognitivo que muestran las personas en juegos p-beauty contest depende no únicamente de su sofisticación cognitiva, sino también de sus creencias sobre la sofisticación de los otros jugadores.\\

Para estudiar de forma más directa la relación entre las creencias y las elecciones de los participantes, Lahav (\citeyear{Lahav}), utilizó un método con \textit{elicited beliefs} (la mejor forma de traducirlo sería “\textit{que hace explícitas las creencias}”) en sesiones experimentales compuestas por 5 periodos de p-beauty contest con hasta 20 participantes. En cada periodo, además de elegir su propio número, se les pidió a los participantes que estimaran cuántos de los otros participantes elegirían un número dentro de cada uno de 10 posibles intervalos en el rango $0-100$ ($0-10$, $11-20$, $21-30$, …, $91-100$) y con dichas estimaciones, se calcularon las creencias de los participantes sobre el número promedio en cada periodo del juego. Al implementar este método, en contraste con investigaciones previas, Lahav concluye que las elecciones no son un reflejo preciso de las creencias de los participantes, pues encuentra diferencias significativas entre el número objetivo computado de acuerdo a las creencias de los participantes acerca de las tiradas de los demás jugadores y el número que de hecho eligen en el juego.\\

Otra investigación con resultados similares fue realizada por Costa-Gomes y Weizsäcker en el \citeyear{Costa-Gomes}, utilizando un método para explicitar creencias en juegos sencillos de 3x3 (donde dos jugadores tienen que elegir entre tres estrategias posibles), en los que es posible llegar al equilibrio del juego mediante razonamiento iterado. En dicha investigación se encontró que en la mayoría de los casos, de acuerdo con el modelo de nivel-k, las creencias registradas por los jugadores acerca de sus oponentes los situaban en el nivel 2, mientras que sus elecciones correspondían al nivel 1. Ya que no parece que las elecciones observadas habrían sido una respuesta óptima ante las creencias elicitadas registraron, este resultado pone en duda que estas últimas sean la base a partir de la cual los participantes emiten sus respuestas. A la luz de estos hallazgos, los autores concluyen que los jugadores basan sus decisiones en reglas de elección que influyen en ambas, creencias y decisiones.\\

Debido a que el presente trabajo de tesis incorpora parte del método de Lahav (\citeyear{Lahav}), para recopilar las creencias de los jugadores acerca de las tiradas de sus oponentes en juegos de p-beauty contest, se enfatizan los siguientes puntos respecto del estudio citado: \\

\begin{enumerate}
\item El método propuesto no permite conocer con exactitud la creencia de los participantes sobre el número objetivo. En su lugar, se obtuvo un cálculo aproximado de esta a partir del número de jugadores que se creyó elegirían un número dentro de cada intervalo, tomando la media de cada intervalo como valor de referencia.\\

\item Con alrededor de 20 personas participando en el juego, parece inverosímil, dada la demanda cognitiva, que los jugadores puedan calcular con precisión el número objetivo derivado de sus creencias acerca de las elecciones del resto de los participantes para emitir su respuesta.\\

\item Se contó con dos grupos control. En uno, no se realizó la explicitación de creencias, y en el otro, dicho procedimiento se realizó después de que los participantes hubieran eligido su número. Comparado la elección de los participantes en los grupos control con el grupo experimental, los resultados indican que solicitar a los participantes que registraran sus creencias no cambia significativamente  su número elegido.\\

\item El último periodo del juego fue el único en el que no se encontraron diferencias significativas entre creencias y elecciones, lo que podría sugerir que dicha discrepancia disminuye con la experiencia.\\

\item Dado que se sabe que los participantes tienden a elegir números cada vez más pequeños en cada periodo y que esto reduce invariablemente la magnitud de cualquier diferencia entre elecciones y creencias (ya que las diferencias entre números más pequeños son, por definición, más pequeñas), Lahav implementó un método de normalización con el que ponderó las diferencias entre las creencias y elecciones de cada participante en cada periodo por el promedio de los números elegidos por todos los participantes en dicho periodo. Sin embargo, como la medida de normalización depende de la tirada de todos los jugadores, la magnitud de la diferencia normalizada es afectada por el nivel cognitivo promedio, y “castiga” (incrementa la magnitud) las diferencias entre creencias y elecciones cuando estas no son tan sofisticadas como las del promedio. Debido a esto, podría no ser la mejor forma de compensar la tendencia al equilibrio.\\
\end{enumerate}

El resultado mencionado en el punto 4 permite cuestionar si la discrepancia entre creencias y elecciones se ve afectada por la experiencia que tienen los participantes en el juego.  En la siguiente sección se revisa el efecto de la experiencia en juegos repetidos de p-beauty contest.

\section{El efecto de la experiencia}

Para estudiar el efecto de la experiencia en juegos repetidos de p-beauty contest, Slonim (\citeyear{Slonim}), realizó sesiones experimentales compuestas por 12 periodos (\textit{i.e. repeticiones del juego}), distribuidos equitativamente en tres Subjuegos con tres jugadores participando en cada uno.\\

En una primera condición, al terminar cada subjuego se reemplazaba a dos de los tres participantes por jugadores nuevos, siendo que sólo un jugador permanecía en el experimento durante los 12 periodos completos. En una segunda condición, los dos participantes retirados al término de cada subjuego eran sustituidos por jugadores con la misma experiencia que el jugador que se mantenía en el juego (es decir, que habían jugado la misma cantidad de periodos). En ambas condiciones, los participantes tenían información acerca del nivel de experiencia de los demás jugadores (el número de periodos jugados).\\

Slonim (\citeyear{Slonim}), reportó que en el primer periodo de los Subjuegos 2 y 3, los jugadores con más experiencia mostraron un mayor nivel cognitivo (es decir, que eligieron números más cercanos a 0) cuando sabían que jugaban contra oponentes con la misma experiencia que ellos, que cuando jugaban contra oponentes que no habían jugado previamente. Este resultado aporta evidencia a favor de que las creencias sobre el nivel cognitivo de los otros jugadores influyen en las elecciones individuales. Por su parte, los jugadores sin experiencia no mostraron diferencias significativas en sus elecciones cuando jugaron con oponentes experimentados o no experimentados.\\

En cuanto al efecto de la experiencia en el desempeño de los jugadores, también se observó que los jugadores experimentados ganan el juego con mayor frecuencia cuando juegan con jugadores sin experiencia. Sin embargo, esta ventaja parece reducirse periodo a periodo, conforme los otros jugadores adquieren experiencia.\\

Un último resultado reportado por Slonim (\citeyear{Slonim}), y probablemente el de mayor relevancia para efectos de la presente tesis, corresponde a un Efecto de Reset en la tendencia a elegir un número cada vez más pequeño entre cada periodo, cuando se inicia un nuevo Subjuego contra participantes novatos. En otras palabras, los jugadores con experiencia eligen números más grandes en los primeros periodos de los Subjuegos 2 y 3, cuando entran nuevos jugadores que se sabe no han jugado anteriormente.\\

Con base en estos hallazgos, y los reportados en las secciones anteriores, se procede a plantear formalmente el objetivo de este trabajo de investigación, así como las estrategias metodológicas empleadas para llevarlo a cabo. 

\section{Objetivo}

El presente trabajo de investigación busca evaluar si la reducción en las diferencias entre creencias y elecciones reportada por Lahav (\citeyear{Lahav}), es verdaderamente el resultado de la experiencia adquirida al participar repetidas veces en un juego, o si por el contrario, es consecuencia de la tendencia reportada en la literatura a elegir números más pequeños en cada repetición del juego.\\

Para responder a esta pregunta, se propone una versión modificada del método de Lahav (\citeyear{Lahav}) que contempla la participación de un grupo más pequeño de jugadores, lo hace más plausible preguntar a los participantes directamente por las creencias que tienen sobre los números que elegirán los demás participantes al reducir la demanda cognitiva de dicha tarea, permitiendo una estimación más precisa del número objetivo que corresponde a las creencias registradas. De esta forma, resulta más verosímil esperar una correspondencia directa entre las elecciones de los participantes y el número objetivo computado de acuerdo a sus creencias.\\

Para atenuar el peso que tiene la tendencia a elegir números cada vez más pequeños en cada periodo, se buscó replicar el efecto de Reset reportado por Slonim (\citeyear{Slonim}), en las elecciones de los jugadores con experiencia. Para ello, algunos participantes permanecieron durante más de un Subjuego y el resto de los jugadores fueron sustituidos por jugadores nuevos en un juego p-beauty contest repetido. Se espera que el efecto de reset opere no sólo en las elecciones, sino también en las creencias del jugador experimentado, lo que permitiría evaluar si la diferencia entre estas sigue reduciéndose, como sugeriría lo reportado por Lahav (\citeyear{Lahav}), aún cuando se abandone la tendencia a reducir las elecciones en el primer periodo del segundo Subjuego jugado. Si a pesar de observarse el efecto de Reset, la diferencia entre sus elecciones y sus creenciaso disminuye, esto aportaría evidencia a favor de que es la experiencia lo que reduce las inconsistencias entre creencias y elecciones.\\

En conjunto, el diseño experimental aquí propuesto permite determinar si las creencias de los jugadores que participan en los dos subjuegos se acercan más a las elecciones reales de los otros jugadores en el segundo subjuego gracias a que cuentan con mayor experiencia.\\

