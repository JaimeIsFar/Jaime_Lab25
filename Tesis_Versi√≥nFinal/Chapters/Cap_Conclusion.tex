% Chapter 1

\chapter{Conclusión} % Main chapter title

\label{Cap_Conclusion} % For referencing the chapter elsewhere, use \ref{Chapter1} 

Se realizó un experimento de p-beauty contest repetido con una variación del método de explicitación de creencias presentado por Lahav, (\citeyear{Lahav}). Los resultados encontrados en el presente estudio confirman el hallazgo principal reportado por Lahav: en un comienzo los participantes eligen números poco consistentes con las creencias reportadas, pero conforme avanzan los periodos, van adquiriendo consistencia. El presente trabajo aporta evidencia a favor de la relación experiencia-consistencia, descartando al efecto de suelo como la causa principal de la reducción en las diferencias encontradas entre las elecciones y creencias registradas en cada periodo, mediante la incorporación de un Subjuego 2 donde algunos de los participantes que adquirieron experiencia durante el Subjuego 1 se enfrentaron a nuevos oponentes, generando un efecto de Reset que llevó a los participantes con experiencia a elegir números más grandes, y aún así, consistentes con sus creencias.\\

Además, los resultados encontrados muestran que los participantes no sólo se vuelven más consistentes conforme adquieren experiencia, sino que también comienzan a elegir números que caen por debajo del promedio de sus creencias (lo cual podría sugerir que aprenden a incorporar la multiplicación por $p$ al elegir el número con el que competirán en cada periodo).\\

Todos los juegos de p-beauty contest que se llevaron a cabo en este trabajo de tesis se realizaron con grupos de únicamente tres participantes. Esta decisión obedeció principalmente a mantener la misma cantidad de jugadores con y sin experiencia que reporta Slonim (\citeyear{Slonim}), además de facilitar la explicitación de creencias específicas al contar con grupos pequeños. Se realizaron 10 sesiones experimentales de acuerdo con diseños experimentales similares presentes en la literatura (Ho, Camerer & Weigelt, \citeyear{Ho}, Kocher, Sutter & Wakolbinger, \citeyear{Kocher}, Slonim, \citeyear{Slonim}).\\ %Adicionalmente, la repetición del juego por 4 periodos en cada subjuego permitió detectar aquellos participantes que no tienen una estructura discernible en sus elecciones.\\

La principal fortaleza del presente diseño experimental radica en la inclusión de un segundo subjuego donde nuevos jugadores entraron en el juego, permitiendo con ello:

\begin{itemize}
\item Un análisis de medidas repetidas de las creencias y elecciones del participante A en dos escenarios diferentes (un Subjuego sin experiencia y un Subjuego con experiencia).

\item Una mayor muestra de jugadores sin experiencia.

\item Poner a prueba, a partir del efecto de Reset reportado en las elecciones de jugadores con experiencia al enfrentarse a jugadores sin experiencia, la posibilidad de que los resultados reportados por Lahav (\citeyear{Lahav}), sean un co-producto de la tendencia hacia el equilibrio que se ha observado de manera consistente en juegos repetidos de p-beauty contest.

\item Al enfrentar a los participantes A de cada sesión experimental con distintos grupos de jugadores, se garantiza que el incremento observado en la consistencia entre sus creencias y elecciones no es resultado de la retroalimentación específica que obtiene sobre la conducta de sus oponentes.
\end{itemize}

En conjunto, estos elementos permiten atribuir los resultados obtenidos a la experiencia, lo que tiene implicaciones directas sobre la pregunta de investigación planteada en un inicio, y en general, la literatura en Teoría de Juegos y razonamiento iterado.\\

Lahav (\citeyear{Lahav}), concluye que parece haber poca evidencia en favor de la consistencia entre creencias y elecciones, y que el nivel de sofisticación de las personas no puede estimarse a partir de sus creencias pues estas no parecen reflejarse en sus elecciones. El autor propone como posibles explicaciones una 'anomalía conductual' o el malentendido de las instrucciones, reconociendo la necesidad de más investigaciones. Por su parte, Costa-Gomes y Weizsäcker (\citeyear{Costa-Gomes}), que estudiaron juegos con creencias hechas explícitas en los que los jugadores no recibían retroalimentación y quienes también encontraron que las elecciones de los participantes no eran consistentes con sus creencias, concluyen que no es posible asumir que las elecciones de las personas son dirigidas por las creencias que tienen sobre los otros jugadores, por lo menos en juegos complejos con ausencia de retroalimentación, y que la construcción de expectativas sobre la conducta de otros jugadores implica un proceso de aprendizaje a partir de la retroalimentación recibida en interacciones repetidas.\\

Debido al diseño del experimento de Lahav (\citeyear{Lahav}), resulta difícil asumir que los jugadores tienen expectativas individuales sobre las elecciones de cada jugador, al tener que enfrentarse a un amplio número de ellos (20 jugadores por sesión experimental). Por su parte, aunque en el experimento de Costa-Gomes y Weizsäcker (\citeyear{Costa-Gomes}), los juegos eran de solo dos participantes, se evitó la retroalimentación, lo que no permitió que se crearan expectativas sobre el otro jugador. En contraste, en el presente experimento se trabajó con grupos pequeños de jugadores y se permitió la retroalimentación al final de cada periodo de juego, dando oportunidad a que se crearan expectativas sobre los otros jugadores. Sin embargo, la  posibilidad de formarse expectativas acerca del comportamiento de los otros jugadores no es suficiente para explicar el aumento en la consistencia entre creencias y elecciones por parte de los participantes, pues en el Subjuego 2, donde se introducen nuevos jugadores el incremento observado en la consistencia de las elecciones se mantiene, efecto que sólo puede atribuirse a su experiencia, ya que de acuerdo con lo previamente sugerido por Slonim (\citeyear{Slonim}), adquirir experiencia acelera la tasa de aprendizaje en un juego, o en juegos similares, tal y como se vio  reflejado en sus resultados, donde luego del primer subjuego, la convergencia al equilibrio fue más rápida y los jugadores con experiencia mantuvieron una mayor ventaja en el juego, aun jugando con personas nuevas. En palabras de Ho, Camerer & Weigelt (\citeyear{Ho}), los jugadores “aprenden a aprender”.\\

El efecto de la experiencia se distingue del efecto de la formación de expectativas sobre los otros jugadores y sus estrategias, porque los cambios en la consistencia se mantienen aun cuando los demás jugadores cambian. De esta forma, aún si las expectativas que se tiene sobre los nuevos jugadores fueran similares a las que se tenían sobre los jugadores anteriores (en el periodo equivalente), y resultaran ser erróneas respecto de las elecciones observadas, esto no compromete a la consistencia que se registra entre sus propias creencias y elecciones, porque se ha adquirido experiencia sobre el juego en sí mismo, sus reglas y la forma en que las personas las aprenden. Esto se alínea con las conclusiones planteadas por Costa-Gomes & Weizsäcker (\citeyear{Costa-Gomes}), acerca de la influencia que tiene la situación (el contexto, el juego) sobre la formación simultánea de creencias y elecciones, que remplaza el supuesto de que la situación influye primeramente en las creencias, que posteriormente permean las elecciones.\\

En su revisión de modelos de no-Equilibrio en Teoría de Juegos, Crawford, Costa-Gomes e Iriberri (\citeyear{Crawford}) concluyen que aunque existe evidencia experimental de que las personas se desvían del equilibrio de Nash, lo hacen de forma sistemática, a partir de un componente estructural que se puede modelar; los modelos de nivel-k son los mejores para predecir la conducta en muchos juegos con este componente, pues permiten predecir dichas desviaciones, detectando sus causas y anticipando su frecuencia, proporcionando respuestas a distintos problemas empíricos. El presente experimento replicó el efecto de Reset observado por Slonim (\citeyear{Slonim}), siendo dicho resultado una evidencia importante a favor de los modelos de nivel-k (que establecen que las personas anclan sus creencias de manera intuitiva de acuerdo con el juego, ajustándolas mediante respuestas óptimas iteradas), en tanto que tras la introducción de nuevos jugadores (sin experiencia; en niveles cognitivos 0 ó 1), los jugadores experimentados se alejan del equilibrio de Nash para responder de forma óptima (a la luz de sus nuevos contrincantes; eligiendo números que corresponderían con niveles cognitivos 1 ó 2). En otras palabras, este tipo de resultados confirman la idea de que los jugadores experimentados son racionales, pero pueden tener  creencias sobre los otros jugadores que impliquen asumir que estos no lo son.\\
